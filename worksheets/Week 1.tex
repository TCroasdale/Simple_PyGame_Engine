\documentclass[11pt]{report}

\usepackage{listings}
\usepackage{color}

\definecolor{dkgreen}{rgb}{0,0.6,0}
\definecolor{gray}{rgb}{0.5,0.5,0.5}
\definecolor{mauve}{rgb}{0.58,0,0.82}

\lstset{frame=tb,
  language=Python,
  aboveskip=3mm,
  belowskip=3mm,
  showstringspaces=false,
  columns=flexible,
  basicstyle={\small\ttfamily},
  numbers=left,
  numberstyle=\tiny\color{gray},
  keywordstyle=\color{blue},
  commentstyle=\color{dkgreen},
  stringstyle=\color{mauve},
  breaklines=true,
  breakatwhitespace=true,
  tabsize=3
}


\begin{document}

\chapter{Introduction}
Throughout the next 8 weeks we will be creating a game! Hopefully you'll all learn a lot about python.

The first step is to learn about the program we'll be using, there will be a lot of code written that you won't understand. Don't worry! hopefully by the end of it you'll have a better understanding of the ins and outs of python and you might be able to see how it all works.

The first file to understand is adventuregame.py 
it looks like this:
\begin{lstlisting}
import pygame
from pygame.locals import *

from Engine.scenenode import *
from Engine.texturemanager import *
from Engine.inputmanager import *
from Engine.object import *
from player import *
from game import *


class AdventureGame(Game):

    # This functions will create objects and add them to the scene.
    def create_scene(self):
        return

if __name__ == "__main__":
    AdventureGame.start_game( AdventureGame())
\end{lstlisting}
On lines 1 to 9 we are importing code from other python files to use within this one. for example, importing pygame allows us to use the pygame library which is the backbone of the entire game.

Lines 12 to 16 define something called a class (which in this case is called AdventureGame) and within this class it creates a function called create_scene. this function is called to add objects to the scene which will then be drawn to the screen.

\end{document}